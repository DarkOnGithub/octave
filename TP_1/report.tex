\documentclass[a4paper,11pt]{article}
\usepackage{graphicx}
\usepackage{subcaption}
\usepackage{float}
\usepackage{xcolor}
\usepackage{array}
\usepackage[T1]{fontenc}
\usepackage{amsfonts}
\usepackage{amsmath}
\usepackage[margin=2.5cm]{geometry}
\usepackage[hidelinks]{hyperref}
\usepackage{titling}
\usepackage{fancyhdr}
\usepackage{enumitem}

\pagestyle{fancy}
\fancyhf{} 
\renewcommand{\headrulewidth}{0pt}
\renewcommand{\footrulewidth}{0.4pt}
\fancyfoot[L]{Compte-rendu 1 - Octave} 
\fancyfoot[R]{Page \thepage}

\title{
\vspace{6cm}
\rule{0.65\linewidth}{3pt}\\[0.7em]
\textbf{Compte-rendu 1}\\[0.5em]
\rule{0.65\linewidth}{1pt}
}

\author{
    \textbf{Tom Huynh} \\
    \texttt{tom.huynh}
}

\date{\today}

\begin{document}

\maketitle\thispagestyle{empty}
\newpage



\section{Exercice 1}
On étudie les trois formes bilinéaires symétriques sur $E = \mathbb{R}^2$ définies pour tout $(u,v) \in E^2$ par:
\begin{align}
    \varphi_A(u,v) = {}^t X_1 \cdot A \cdot X_2, \quad
    \varphi_B(u,v) = {}^t X_1 \cdot B \cdot X_2 \quad \text{et} \quad
    \varphi_C(u,v) = {}^t X_1 \cdot C \cdot X_2
\end{align}
où $X_1$ et $X_2$ sont les matrices colonnes constituées des coordonnées de $u$ et de $v$ dans la base canonique.

L'objectif est de déterminer si ces formes bilinéaires symétriques sont des produits scalaires, c'est-à-dire si elles sont définies positives. Les matrices extraites du fichier \texttt{Ex1.mat} sont les suivantes :
\begin{equation*}
    A = \begin{pmatrix} 1,7500 & 0,4330 \\ 0,4330 & 1,2500 \end{pmatrix}, \quad
    B = \begin{pmatrix} 0,5000 & 0,8660 \\ 0,8660 & 1,5000 \end{pmatrix}, \quad
    C = \begin{pmatrix} 0,5000 & 1,5000 \\ 1,5000 & 0,5000 \end{pmatrix}
\end{equation*}

Pour analyser ces formes, on considère le vecteur $u(t) = (\cos(t), \sin(t))$ pour tout $t \in [0, 2\pi]$. Les figures~\ref{fig:phi_A},~\ref{fig:phi_B} et~\ref{fig:phi_C} présentent respectivement les graphes de ces fonctions pour les matrices $A$, $B$ et $C$.

\begin{figure}[!ht]
    \centering
    \begin{subfigure}{0.44\textwidth}
        \centering
        \includegraphics[width=\linewidth]{ex_1_image_1.png}
        \caption{Graphe de $\varphi_A(u(t), u(t))$}
        \label{fig:phi_A}
    \end{subfigure}
    \hfill
    \begin{subfigure}{0.44\textwidth}
        \centering
        \includegraphics[width=\linewidth]{ex_1_image_2.png}
        \caption{Graphe de $\varphi_B(u(t), u(t))$}
        \label{fig:phi_B}
    \end{subfigure}

    \begin{subfigure}{0.55\textwidth}
        \centering
        \includegraphics[width=\linewidth]{ex_1_image_3.png}
        \caption{Graphe de $\varphi_C(u(t), u(t))$}
        \label{fig:phi_C}
    \end{subfigure}
    \caption{Visualisation des formes bilinéaires pour les matrices $A$, $B$ and $C$.}
    \label{fig:exercice1}
\end{figure}


Pour la matrice $A$, la fonction $\varphi_A(u(t), u(t))$ reste strictement positive sur tout l'intervalle $[0, 2\pi]$, suggérant que $\varphi_A$ pourrait être un produit scalaire. En revanche, le graphe de la matrice $B$ va jusqu'à zéro tandis que celui de la matrice $C$ descend jusqu'à $-1$, ce qui confirme que ces deux formes ne sont pas définies positives.

On remarque également que sur chaque graphique, les valeurs propres de la matrice correspondent exactement aux extremums (minimum et maximum) de la fonction. Par exemple, pour la matrice $A$, le maximum est $2$ et le minimum est $1$.

Pour confirmer cette analyse visuelle, on procède à la diagonalisation des trois matrices. Une forme bilinéaire symétrique est un produit scalaire si et seulement si elle est définie positive, c'est-à-dire si toutes les valeurs propres de la matrice associée sont strictement positives.

La diagonalisation confirme les observations graphiques. Les valeurs propres obtenues sont les suivantes :
\begin{itemize}
    \item Pour la matrice $A$ : $2$ et $1$. Les deux étant strictement positives, $\varphi_A$ est un produit scalaire.
    \item Pour la matrice $B$ : $2$ et environ $-5,55 \times 10^{-17}$. La présence d'une valeur propre non strictement positive (nulle / légèrement négative en raison de la précision numérique) montre que $\varphi_B$ n'est pas un produit scalaire.
    \item Pour la matrice $C$ : $2$ et $-1$. La valeur propre négative confirme que $\varphi_C$ n'est pas un produit scalaire.
\end{itemize}

En conclusion, l'étude des valeurs propres permet de caractériser la nature d'une forme bilinéaire symétrique, confirmant que seule la matrice $A$ définit un produit scalaire.


\section{Exercice 2 : Étude des endomorphismes symétriques}

On étudie les endomorphismes $f$, $g$ et $h$ de $E = \mathbb{R}^2$ définis par les matrices $A$, $B$ et $C$ dans la base canonique. L'objectif est d'observer l'image du cercle unité $\mathcal{C}$ par ces transformations.

L'image du cercle unité par un endomorphisme symétrique est une ellipse. Les figures~\ref{fig:image_f},~\ref{fig:image_g} et~\ref{fig:image_h} illustrent ces transformations, avec la représentation des vecteurs $\lambda_1 \varepsilon_1$ et $\lambda_2 \varepsilon_2$, où $(\varepsilon_1, \varepsilon_2)$ est la base propre et les $\lambda_i$ les valeurs propres.

\begin{figure}[!ht]
    \centering
    \begin{subfigure}{0.44\textwidth}
        \centering
        \includegraphics[width=\linewidth]{ex_2_image_1.png}
        \caption{Image par $f$ (matrice $A$)}
        \label{fig:image_f}
    \end{subfigure}
    \hfill
    \begin{subfigure}{0.44\textwidth}
        \centering
        \includegraphics[width=\linewidth]{ex_2_image_2.png}
        \caption{Image par $g$ (matrice $B$)}
        \label{fig:image_g}
    \end{subfigure}

    \begin{subfigure}{0.55\textwidth}
        \centering
        \includegraphics[width=\linewidth]{ex_2_image_3.png}
        \caption{Image par $h$ (matrice $C$)}
        \label{fig:image_h}
    \end{subfigure}
    \caption{Images du cercle unité par les endomorphismes $f$, $g$ et $h$.}
    \label{fig:exercice2}
\end{figure}

La diagonalisation des matrices permet d'obtenir les valeurs propres $\lambda_i$ et les vecteurs propres associés $\varepsilon_i$. On remarque un comportement standard des solveurs numériques: la matrice de passage $P$ retournée est systématiquement normalisée, c'est-à-dire que chaque vecteur propre $\varepsilon_i$ est de norme euclidienne égale à 1. Pour la matrice $A$, le solveur retourne la matrice de passage suivante (arrondie à $10^{-4}$) :
\begin{equation*}
    P_A = \begin{pmatrix} -0,8660 & 0,5000 \\ -0,5000 & -0,8660 \end{pmatrix}
\end{equation*}
On vérifie bien que $\|\varepsilon_1\| = \sqrt{(-0,8660)^2 + (-0,5000)^2} = \sqrt{0,749956 + 0,2500} \approx 1$. Cette normalisation est importante car elle permet de s'assurer que les axes de l'ellipse ont des longueurs exactement égales aux valeurs propres $\lambda_i$.

De plus, comme les matrices $A, B$ et $C$ sont symétriques, leurs vecteurs propres associés à des valeurs propres distinctes sont théoriquement orthogonaux. Numériquement, on vérifie l'orthonormalité de la base propre en calculant le produit $P^T P$. Pour la matrice $A$, Octave retourne :
\begin{equation*}
    P^T P = \begin{pmatrix} 1,0000 & 1,4874 \times 10^{-17} \\ 1,4874 \times 10^{-17} & 1,0000 \end{pmatrix} \approx I
\end{equation*}
Pour la matrice $B$, le résultat est même exactement la matrice identité à la précision d'affichage près, tandis que pour $C$, le terme hors-diagonale est de l'ordre de $-1,0147 \times 10^{-17}$. Ces résultats confirment que la base $(\varepsilon_1, \varepsilon_2)$ est orthonormée.

Les figures~\ref{fig:image_f},~\ref{fig:image_g} et~\ref{fig:image_h} illustrent ces transformations. Sur chaque graphique, sont représentés:
\begin{itemize}
    \item Le cercle unité $\mathcal{C}$ (en pointillé noir).
    \item L'image du cercle par l'endomorphisme (en bleu).
    \item Les vecteurs propres pondérés par leurs valeurs propres $\lambda_1 \varepsilon_1$ (rouge) et $\lambda_2 \varepsilon_2$ (vert).
\end{itemize}

Géométriquement, les axes de chaque ellipse sont dirigés selon les vecteurs propres de la matrice associée, leurs longueurs respectives étant égales aux valeurs propres (en valeur absolue). La nature orthonormée de la base propre garantit ainsi que les axes de l'ellipse soient perpendiculaires.

Les résultats observés permettent de tirer plusieurs conclusions :
\begin{itemize}
    \item Pour la matrice $A$ (Figure~\ref{fig:image_f}), l'image est une ellipse dont les rayons, valant 2 et 1, sont définis par les valeurs propres.
    \item Pour la matrice $B$ (Figure~\ref{fig:image_g}), l'ellipse est aplatie au point de ressembler à un segment. C'est la conséquence directe de la valeur propre quasi nulle ($\approx -5,55 \times 10^{-17}$) : le cercle unité est projeté sur la droite dirigée par le vecteur propre $\varepsilon_1$. Géométriquement, l'endomorphisme $g$ agit comme une projection sur un sous-espace vectoriel de dimension 1 (la droite $\text{Vect}(\varepsilon_1)$).
    \item Pour la matrice $C$ (Figure~\ref{fig:image_h}), la valeur propre négative ($\lambda_2 = -1$) fait un changement de sens dans la transformation.
\end{itemize}

En conclusion, l'image du cercle unité permet de voir l'action de la matrice : elle étire ou contracte le cercle selon ses valeurs propres, le transformant en une ellipse orientée par ses vecteurs propres.


\section{Exercice 3 : Représentation du cercle unité pour une norme induite}

La forme bilinéaire symétrique $\varphi_A$ associée à $A$ définit un produit scalaire sur $E = \mathbb{R}^2$ :
\begin{equation}
    \varphi_A(u,v) = {}^t X_1 \cdot A \cdot X_2
\end{equation}
où $X_1$ et $X_2$ sont les coordonnées de $u$ et $v$ dans la base canonique. Ce produit scalaire induit une norme sur $E$, notée $\|\cdot\|_A$ :
\begin{equation}
    \|u\|_A = \sqrt{\varphi_A(u,u)} = \sqrt{{}^t X \cdot A \cdot X}
\end{equation}

L'objectif est de représenter le cercle unité vis-à-vis de cette norme, c'est-à-dire l'ensemble $\mathcal{C}_A = \{u \in E, \|u\|_A = 1\}$.

\begin{figure}[H]
    \centering
    \includegraphics[width=0.8\linewidth]{ex_3_image.png}
    \caption{Représentation du cercle unité pour la norme $\|\cdot\|_A$ comparé au cercle unité.}
    \label{fig:exercice3}
\end{figure}

Pour ce faire, on travaille dans la base propre $\mathcal{B}' = (\varepsilon_1, \varepsilon_2)$ de $A$. Soit $X' = (x', y')$ les coordonnées d'un vecteur $u$ dans cette base. L'expression de la forme quadratique devient alors:
\begin{equation}
    \varphi_A(u,u) = {}^t X' \cdot D \cdot X' = \lambda_1 (x')^2 + \lambda_2 (y')^2
\end{equation}
L'équation du cercle unité $\mathcal{C}_A$ dans la base propre est donc $\lambda_1 (x')^2 + \lambda_2 (y')^2 = 1$. On peut alors décrire cet ensemble par :
\begin{equation}
    x' = \frac{\cos(t)}{\sqrt{\lambda_1}} \quad \text{et} \quad y' = \frac{\sin(t)}{\sqrt{\lambda_2}} \quad \text{pour } t \in [0, 2\pi]
\end{equation}

Pour représenter $\mathcal{C}_A$ dans la base canonique, on effectue le changement de base $X = P \cdot X'$, où $P$ est la matrice de passage composée des vecteurs propres.

Le graphique de la Figure~\ref{fig:exercice3} montre que le cercle unité pour la norme $\|\cdot\|_A$ est une ellipse dont les axes sont alignés avec les vecteurs propres de $A$. Les demi-longueurs des axes sont inversement proportionnelles aux racines carrées des valeurs propres ($1/\sqrt{\lambda_1} = 1/\sqrt{2} \approx 0,707$ et $1/\sqrt{\lambda_2} = 1/\sqrt{1} = 1$). 

On observe une différence fondamentale avec l'exercice 2 : alors que l'image du cercle unité par l'endomorphisme $f$ (de matrice $A$) dilatait le cercle selon les valeurs propres, le cercle unité pour la norme $\|\cdot\|_A$ subit une contraction sur les axes associés aux valeurs propres supérieures à 1.

En conclusion, la représentation du cercle unité pour une norme induite montre que la géométrie de l'espace est contractée selon les directions du produit scalaire, montrant la relation inverse entre les valeurs propres et les demi-longueurs des axes de l'ellipse.

\end{document}
